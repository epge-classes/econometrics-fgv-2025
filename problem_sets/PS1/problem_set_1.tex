\documentclass[11pt]{article}
\usepackage[letterpaper,margin=2cm]{geometry}
\usepackage[english]{babel}
\usepackage[utf8]{inputenc}
\usepackage[T1]{fontenc}
\linespread{1.3}
\parskip=12pt
\parindent=0pt
\usepackage{enumerate}
\usepackage{amsfonts}
\usepackage{amsmath}
\usepackage{amsfonts}
\usepackage{graphicx}
\usepackage[section]{placeins}
\usepackage{amsthm}
\usepackage{amssymb}
\usepackage{mathpazo}

\usepackage[dvipsnames]{xcolor}
\definecolor{slateblue}{RGB}{45, 62, 80}
\definecolor{FGVBlue}{RGB}{0, 114, 188}

% Defining the question styles
\theoremstyle{definition}
\newtheorem{prob}{Problem}

% Custom commands
\newcommand{\E}{\mathbb{E}}
\newcommand{\Var}{\mathrm{Var}}
\newcommand{\Prob}{\mathbb{P}}

% Command to start a new section with the word "Problem" and enumerate automatically
\newcounter{problem}
\renewcommand{\theproblem}{\arabic{problem}}

\newcommand{\problem}{
	\stepcounter{problem}
	\section*{Problem \theproblem}
	\addcontentsline{toc}{section}{Problem \theproblem}
}

\begin{document}
	\begin{center}
		{\LARGE{\textbf{Problem Set I}}}\\
		\vspace{0.2cm}
		Econometrics I - \textcolor{FGVBlue}{FGV EPGE}\\
		Instructor: Raul Guarini Riva \\
		TA: Taric Latif Padovani
	\end{center}

\problem
Let $X$ be a scalar random variable with density $f(x)$. Let $K(\cdot)$ be a symmetric second-order kernel. For a given point $x$ in the interior of the support of $f(\cdot)$, define the density estimator as
\[\hat{f}_n(x) \equiv \frac{1}{nh} \sum_{i=1}^n K\left(\frac{X_i - x}{h}\right),\]
where \(h > 0\) is a bandwidth parameter and \(X_1, \ldots, X_n\) are independent and identically distributed (i.i.d.) random variables with density \(f(\cdot)\).

This exercise will show you how to ensure that $\hat{f}_n(x) \xrightarrow{p} f(x)$ as \(n \rightarrow \infty\), $h \rightarrow 0$, and $nh \rightarrow \infty$. This is the same asymptotic framework as in the slides.

\begin{enumerate}[a)]
	\item Show that $\hat{f}_n(x) \geq 0$ for all $x$, and that $\int_{-\infty}^{\infty} \hat{f}_n(x) dx = 1$ for all $n$.
	\item Assume from now on that $f$ is continuous at $x$. Show that $\mathbb{E}[\hat{f}_n(x)] = f(x) + o(1)$.
	\item Show that $\Var(\hat{f}_n(x)) = \frac{1}{nh}\cdot f(x)R(K) + o\left(\frac{1}{nh}\right)$, where \(R(K) = \int_{-\infty}^{\infty} K^2(u) du\).
	\item Argue that these results imply that $\hat{f}_n(x)$ is consistent for $f(x)$.
	\item Now, assume that $f$ is twice continuously differentiable at $x$. Show that 
	\begin{equation*}
		\mathbb{E}[\hat{f}_n(x)] = f(x) + \frac{h^2}{2}f''(x)R(K) + o(h^2)
	\end{equation*}
	\item Explain in words how the local convexity of $f$ might (or might not) affect this finite-sample bias.
\end{enumerate}

Hint: A very useful resource for this question is Chapter 17 from \textit{Probability and Statistics for Economists} by Bruce Hansen.

\problem
Let $X$ be a continuous random variable with density $f(\cdot)$, which is positive everywhere. Suppose the true regression function is linear, \( m(X = x) = \alpha + \beta x \), and we estimate the function using the Nadaraya-Watson estimator. Assume all regularity conditions you need.

\begin{enumerate}[a)]
	\item Calculate the bias function \( B(x) \).
	\item Suppose \( \beta > 0 \). For which regions is \( B(x) > 0 \) and for which regions is \( B(x) < 0 \)?
	\item Now suppose that \( \beta < 0 \) and re-answer the question.
	\item Can you intuitively explain why the Nadaraya-Watson estimator is positively or negatively biased in these regions?
\end{enumerate}

\problem
This is an empirical question based on Karlan and Zinman (2008, Econometrica). You will find the paper online on the class Github repo. The data used in the paper is also available there.

\begin{enumerate}[a)]
	\item What is the main research question in the paper? What is the most striking finding? Answer in just a few sentences.
	\item Your goal will be to estimate $\mathbb{P}(applied = 1 | offer4 = x)$. Note that \texttt{applied} is a binary variable, while \texttt{offer4} is continuous. These are the only variables you will need. Notice that 
	\[
	\mathbb{P}(applied = 1 | offer4 = x) = \mathbb{E}[applied | offer4 = x]
	\]

	\item Use a Gaussian kernel to estimate this probability and show a plot of your estimates for a range of values of \texttt{offer4}. Do this with three different bandwidths: 
	\begin{itemize}
		\item Silverman’s rule of thumb: $h = 1.06 \cdot \hat{\sigma} \cdot n^{-1/5}$, where $\hat{\sigma}$ is the standard deviation of \texttt{offer4} and $n$ is the number of observations;
		\item A value much \textit{smaller} than that;
		\item A value much \textit{larger} than that;
	\end{itemize}
	\item Do the same with the Epanechnikov kernel, using the same bandwidths as before.
	\item Compare the results of the two kernels qualitatively.
\end{enumerate}
	
\end{document}