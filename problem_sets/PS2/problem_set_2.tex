\documentclass[11pt]{article}
\usepackage[letterpaper,margin=2cm]{geometry}
\usepackage[english]{babel}
\usepackage[utf8]{inputenc}
\usepackage[T1]{fontenc}
\linespread{1.3}
\parskip=12pt
\parindent=0pt
\usepackage{enumerate}
\usepackage{amsfonts}
\usepackage{amsmath}
\usepackage{amsfonts}
\usepackage{graphicx}
\usepackage[section]{placeins}
\usepackage{amsthm}
\usepackage{amssymb}
\usepackage{mathpazo}

\usepackage[dvipsnames]{xcolor}
\definecolor{slateblue}{RGB}{45, 62, 80}
\definecolor{FGVBlue}{RGB}{0, 114, 188}
\usepackage[colorlinks=true, urlcolor=FGVBlue, linkcolor=FGVBlue, citecolor=FGVBlue]{hyperref}

% Defining the question styles
\theoremstyle{definition}
\newtheorem{prob}{Problem}

% Custom commands
\newcommand{\E}{\mathbb{E}}
\newcommand{\Var}{\mathrm{Var}}
\newcommand{\Prob}{\mathbb{P}}

% Command to start a new section with the word "Problem" and enumerate automatically
\newcounter{problem}
\renewcommand{\theproblem}{\arabic{problem}}

\newcommand{\problem}[1]{
	\stepcounter{problem}
	\section*{Problem \theproblem{} -- (points: #1)}
}

\begin{document}
	\begin{center}
		{\LARGE{\textbf{Problem Set II}}}\\
		\vspace{0.2cm}
		Econometrics I - \textcolor{FGVBlue}{FGV EPGE}\\
		Instructor: Raul Guarini Riva \\
		TA: Taric Latif Padovani
	\end{center}

\problem{1}
In this problem, we will play with different concepts.

\begin{enumerate}[a)]
	\item Consider the following process:
	\begin{equation*}
		y_t = \cos(t + \theta), \forall t \in \mathbb{R}
	\end{equation*}
	where $\theta$ is a discrete random variable, distributed with uniform probability over the set $[0, \pi/2, \pi, 3\pi/2]$.

	Show that this process is covariance stationary (or weakly stationary).

	\item Show that this process is \textbf{not} strictly stationary.
	\item Now, we consider a different process. Let $z_t$ be such that:
	\begin{equation*}
		z_t = \begin{cases}
			1,\quad \forall t & \text{with probability } 0.5 \\
			0, \quad \forall t & \text{with probability } 0.5
		\end{cases}
	\end{equation*}
	Argue that this process is weakly stationary by computing its first two moments and its autocovariance function. Now, argue that the time-series mean of \textit{any} realized path will \textit{never} converge towards the mean of $z_t$.
\end{enumerate}

\problem{2}
Consider the following AR(2) process:
\begin{equation*}
	(I - 1.1L + 0.18L^2)y_t = \varepsilon_t
\end{equation*}
where $\varepsilon_t$ is a white noise process. Is this process stationary? If so, compute its mean, its variance, and its whole autocovariance function.

\problem{1}
This question is a "warm-up" exercise that will be useful for the following problem.

Consider the polynomial $f(x) = a_p x^p + a_{p-1} x^{p-1} + \cdots + a_1 x + a_0$, where $a_i \in \mathbb{R}$ for all $i = 0, 1, \ldots, p$ and $a_p \neq 0$. Further assume that $a_0 \neq 0$. Let $\lambda$ be any root for this polynomial.

\begin{enumerate}[a)]
	\item Argue that $\lambda \neq 0$;
	\item Consider another polynomial $g(x) = a_p + a_{p-1} x + \cdots + a_1 x^{p-1} + a_0 x^p$. Show that $1/\lambda$ is a root of $g(.)$;
	\item Argue that if $f(.)$ has all its roots outside the unit circle, then $g(.)$ has all its roots inside the unit circle.
\end{enumerate}

\problem{3}
In this question, you will prove a claim we made in the slides. Consider a general $AR(p)$ model:
\begin{equation*}
y_t = \mu + \phi_1 y_{t-1} + \phi_2 y_{t-2} + \cdots + \phi_p y_{t-p} + \varepsilon_t	
\end{equation*}
where $\varepsilon_t$ is a white noise process. Define the operator $\Phi(L) = I - \phi_1 L - \phi_2 L^2 - \cdots - \phi_p L^p$. We assume that all roots of its associated polynomial $f(x) = 1 - a_1x - a_2x^2 - \cdots - a_px^p$ lie outside the unit circle. As we saw in the slides, this implies that there exists an operator $\Psi(L) = \psi_0 + \psi_1 L + \psi_2 L^2 + \cdots$ such that $\Psi(L)\Phi(L) = I$.

In that case, we can write $y_t = \Psi(L)\mu + \Psi(L)\varepsilon_t$.

\begin{enumerate}
	\item Compute $\Psi(L)\mu$ explicitly as a function of $\mu$, and $(\phi_1, ..., \phi_p)$.
	
	\textit{Hint}: what happens when you apply $\Phi(L)$ to a constant?

	\item We argued in class that the invertibility condition for $\Phi(L)$ was enough to ensure that this process is stationary because its MA($\infty$) representation is stationary. We will now work in steps to prove this claim. 
	
	Recall that the MA representation will be stationary if $\sum\limits_{i=0}^{\infty}\psi_i^2 < \infty$. We need to prove this result. Our approach will be directly characterizing $\psi_i$ as a function of $(\phi_1, ..., \phi_p)$.

	By definition, we have that:
	\begin{equation*}
		(1 - \phi_1 L - \phi_2 L^2 - \cdots - \phi_p L^p)(\psi_0 + \psi_1 L + \psi_2 L^2 + \cdots) = 1
	\end{equation*}

	After performing these infinite multiplications, constant terms, terms that depend on $L$, terms that depend on $L^2$, terms that depend on $L^3$, and so on. Any terms that depend on powers of $L$ should be identically zero, by definition.

	Show that 
	\begin{equation}
		\label{eq:recurrence}
		\psi_k = \phi_1 \psi_{k-1} + \phi_2 \psi_{k-2} + ... + \phi_p \psi_{k-p}, \qquad \forall k > p.
	\end{equation}

	\item Recognize that \eqref{eq:recurrence} is a linear, homogenous, $p$-order difference equation in the $\psi$'s. Find its characteristic polynomial and explain why all its roots lie inside the unit circle.
	
	\item It is a classic mathematical fact that all solutions to \eqref{eq:recurrence} are linear combinations of at most $p$ terms of form $m(k)\cdot r^k$ where $m(.)$ is a polynomial in $k$ of order at most $p$, and $r$ is a root of the characteristic polynomial. If such a fact is too obscure, \href{https://wwz.unibas.ch/fileadmin/user_upload/wwz/00_Professuren/Kleiber_Oekonometrie_und_Statistik/Downloads/Topics/Skript008_DifferenceEquations.pdf}{click here}.
	
	Using this fact, explain why $\sum\limits_{i=0}^{\infty}\psi_i^2 < \infty$. 
	
	\emph{Hint}: think about the different tests for the convergence of infinite series.
\end{enumerate}

\problem{3}
In this question, you will analyze two common price indexes for the Brazilian economy: IPCA and IGP-M. The data for this question is contained on the \texttt{ipca\_igpm.csv} file. You have access to the monthly time series of the month-over-month changes for these price indexes.

\begin{enumerate}[a)]
	\item What are the main conceptual differences between IPCA and IGP-M? What are they measuring?
	\item Create a time-series plot with the two series.
	\item Assume both series are stationary. Compute their mean, variance, and autocovariances up to lag 24. Which one displays the most volatility?
	\item Plot the autocorrelation function of both series. How would you describe the differences in persistence between them?
	\item Given your answer in item a), are the results from items b) and c) in line with your expectations? Discuss your intuition.
\end{enumerate}


\end{document}