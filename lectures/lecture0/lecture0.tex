% Customized by Raul Riva for the Econometrics course at FGV EPGE in 2025
\documentclass[aspectratio=169, xcolor=dvipsnames, 12pt]{beamer}
\usetheme{moloch}
\molochset{
  progressbar = frametitle,
  block = fill
}
% Use beamer's page number in head/foot template
\setbeamertemplate{footline}[frame number]
\setbeamersize{text margin left=0.5cm,text margin right=0.5cm}

% from https://tex.stackexchange.com/a/441705/36296
\makeatletter
\addtobeamertemplate{date}{\centering}{}
\addtobeamertemplate{institute}{\centering}{}
\patchcmd{\beamer@@tmpl@title}{\raggedright}{\centering}{}{}
\patchcmd{\beamer@@tmpl@author}{\raggedright}{\centering}{}{}
\setlength{\moloch@titleseparator@linewidth}{2pt}
\setlength{\moloch@progressinheadfoot@linewidth}{1pt}
\makeatother

% Packages
\usepackage{graphicx}
\usepackage{amsmath, amssymb}
\usepackage{subcaption}
\usepackage{booktabs}
\usepackage{xcolor} % For colored bullets
\usepackage{appendixnumberbeamer}

\title[]{Lecture 0: Class Setup}
\author{\textbf{Raul Riva}}
\institute{\scriptsize \textcolor{FGVBlue}{FGV EPGE}}
\date{October, 2025}

% ----- Custom commands -----
\newtheorem{prop}[theorem]{Proposition}

% ----- Defining colors -----
\usepackage[dvipsnames]{xcolor}
\definecolor{slateblue}{RGB}{45, 62, 80}
\definecolor{FGVBlue}{RGB}{0, 114, 188}
\setbeamercolor{palette primary}{bg=slateblue, fg=white}
\setbeamercolor{title separator}{bg=FGVBlue, fg=FGVBlue}
%\setbeamercolor{background canvas}{bg=white}
\setbeamercolor{progress bar}{fg=FGVBlue,bg=white}

% Hyperlinks
\usepackage{hyperref}
\hypersetup{
    colorlinks=true,
    linkcolor=FGVBlue,
    urlcolor=FGVBlue,
    citecolor=FGVBlue
    }
    
% ----- Bibliography -----
%\usepackage[backend=biber,style=authoryear]{biblatex}
%\addbibresource{}
% \DeclareFieldFormat{labelyear}{#1}
% \DeclareFieldFormat{shorthand}{#1}
% \DeclareFieldFormat{extraalpha}{#1}
% \DeclareNameFormat{default}{\namepartfamily}
% \AtEveryCitekey{\color{FGVBlue}}

%---------------- Document Start ----------------%
\begin{document}
%---------------- Title Slide ----------------%
\begin{frame}[plain]
  \titlepage
\end{frame}

%---------------- Instructor Info ----------------%
\begin{frame}{Course Information}
  \begin{itemize}
    \item \textbf{Course}: Econometrics I (Part I);
    \item \textbf{Instructor}: Raul Guarini Riva — \texttt{raul.riva@fgv.br};
    \item \textbf{Teaching Assistant}: Taric Latif — \texttt{tariclatif@gmail.com};
    \item \textbf{Class Schedule}: Tuesdays and Thursdays, 11:00–13:00;
    \item \textbf{Office Hours}: Fridays, 9:00-10:00;
    \item \textbf{TA Session Schedule}: check with Taric;
    \item GitHub repository: \url{https://github.com/rgriva/econometrics-fgv-2025}
    \item After class, you should review the online syllabus in detail;
  \end{itemize}
  \vspace{1em}
  \centering
  \alert{We will use Github for course materials, assignments, and so on.}
\end{frame}

%---------------- Motivation ----------------%
\begin{frame}{Why Should You Take This Class Seriously?}
  \begin{itemize}
    \item Most students will engage in empirical research across diverse fields of Economics:
      \begin{itemize}
        \item Labor, Development, Urban, Finance, Health, IO, even Macro...!
        \item Obviously important if you want to do Econometrics;
      \end{itemize}
    \item So far, most of what you learned is about \textbf{cross-sectional data} and \textbf{OLS};
    \item We go beyond cross-sectional data to tackle: \textbf{Time Series} and \textbf{Panel Data};
    \item Time Series data is your bread and butter in Macro, Finance, and many other fields;
    \item Panel Data methods (fixed effects, diff-in-diff, ...) are everywhere in empirical research;
    \item We will also cover a bit of non-parametric methods and bootstrap techniques;
  \end{itemize}
\end{frame}

%---------------- Learning Goals ----------------%
\begin{frame}{Learning Goals (by Week 5)}
  After the first half of the course, you should be able to:

  \begin{enumerate}\itemsep1em
    \item Understand and implement non-parametric kernel-based methods;
    \item Apply bootstrap methods for i.i.d. data;
    \item Analyze time series data:
      \begin{itemize}
        \item Concepts: stationarity, ergodicity, persistence, autocorrelation, ...
        \item Estimation and forecasting with ARMA models;
        \item Asymptotic distribution of the OLS estimator under autocorrelation;
      \end{itemize}
    \item Implement GMM estimation under heteroskedasticity and autocorrelation;
    \begin{itemize}
      \item Super relevant for Panel Data methods;
      \item Also very useful to estimate structural models;
    \end{itemize}
  \end{enumerate}
\end{frame}

\begin{frame}{Books? Slides?}
  \begin{itemize}\itemsep1em
    \item I will provide slides for the lectures;
    \item These slides will be mostly based on two books + a famous reference here and there:
      \begin{itemize}
        \item \textbf{Econometrics}, by Bruce Hansen (hello, old friend!);
        \item \textbf{Time Series Analysis}, by James Hamilton (very comprehensive, also very long);
        \item \textbf{Stochastic Limit Theory}, by James Davidson (seriously dark magic)
      \end{itemize}
    \item I will highlight the recommended readings in the slides, and reading the books is \textbf{mandatory}.
    \item Slides are not a substitute for reading the books;
    \item I will make slides available on Github just before class;
  \end{itemize}
\end{frame}

%---------------- Topics Overview ----------------%
\begin{frame}{\underline{Tentative} Weekly Topics}
  \begin{itemize}
    \item \textbf{Week 1:} Non-parametric estimation + Bootstrap for i.i.d. data;
    \item \textbf{Week 2:}
    \begin{itemize}
      \item The structure of time series data, stationarity, ergodicity, persistence, autocorrelation, ...;
      \item ARMA models + the Box-Jenkins methodology + Wold Decomposition;
    \end{itemize}
    \item \textbf{Week 3:} 
    \begin{itemize}
      \item Estimation of ARMA models, MLE estimators...
      \item LLN's and CLT's for dependent data -- focus on using the results;
    \end{itemize}
    \item \textbf{Week 4:} 
    \begin{itemize}
      \item LLN and CLT's for dependent data -- focus on using the results rather than proving every single little thing;
      \item Intro to GMM;
    \end{itemize}
    \item \textbf{Week 5:}
    \begin{itemize}
      \item GMM asymptotic theory;
      \item GMM and HAC estimation + numerical pitfalls;
  \end{itemize}
  \end{itemize}
  \vspace{1em}
  \textit{Note: subject to change.}
\end{frame}

\begin{frame}{Evaluation and Grades}
  \begin{itemize}
    \item Grades for the first and second half of the class are independent;
    \item Final grade = 0.5 $\times$ (Grade on Part I) + 0.5 $\times$ (Grade on Part II);
    \item The first part has 3 problem sets and 1 exam:
    \begin{equation}
      \text{Grade on Part I} = \frac{1}{9} \times \text{PS1} + \frac{1}{9} \times \text{PS2} + \frac{1}{9} \times \text{PS3} + \frac{2}{3} \times \text{Exam}
    \end{equation}
  \end{itemize}
\end{frame}

\begin{frame}{Problem Sets}
\begin{itemize}
  \item You can work in groups of up to 3 students;
  \item Late solutions are not accepted;
  \item There will be both theoretical and empirical questions;
  \item Problem set solutions must be submitted in PDF format through Github;
  \item Create a folder for you group under \texttt{student\_work/name1\_name2\_name3/}. Please upload your work there;
  \item You should deliver both a report answering the questions and the code used to generate the results;
  \item Feel free to use any civilized programming language (Python, R, Julia, Matlab, ...);
  \item No, you cannot use Stata or pre-packaged solutions in any of the languages above -- details will be provided in the problem sets;
\end{itemize}
\end{frame}

\begin{frame}{Attendance}
\begin{itemize}
  \item It is not mandatory at all -- you are adults;
  \item However, if you come, you have to be on time;
  \item 10-minute grace period -- after that, you will not be allowed to enter the classroom;
\end{itemize}
  
\end{frame}
\begin{frame}[standout]
Questions?
\end{frame}

\begin{frame}{One more thing...}
  \begin{center}
    \alert{\textbf{This is really important!}}
  \end{center}
  \begin{itemize}
    \item This is the first time I teach this class;
    \item I need ongoing feedback;
    \item Is there something great about? Tell me! If there something terrible about it? Also tell me!
    \item Don't be shy -- negative feedback will never be punished in any way;
    \item If anything, on the contrary!
  \end{itemize}
\end{frame}

\end{document}